
% introduction here

intro, literature review, etc

The remainder of this paper will be organized as follows.
\cref{sec:review} briefly reviews the Monge-Kantorivich problem, its entropic regularized Sinkhorn problem and solution.
\cref{sec:sinkhorn-algorithm} introduces the Sinkhorn algorithm and its variants to solve the entropic regularized problem.
\cref{sec:sinkhorn-gradient} derives the gradients and Jacobians of the family of Sinkhorn algorithms.
\cref{sec:wasserstein-barycenter} discusses the Wasserstein Barycenter and Wasserstein Dictionary Learning
problem.
\cref{sec:barycenter-jacobian} shows the gradient of Sinkhorn-like algorithms to solve the Wasserstein Barycenter problem.
\cref{sec:wdl} considers the Wasserstein Dictionary Learning problem,
which is built on top of the Barycenter problem and ``learns'' the latent topics.
\cref{sec:wig-model} finally presents the Wasserstein Index Generation model for automatically time-series index generation,
based on Wasserstein Dictionary Learning.
\cref{sec:wig-package} provides some code snippets of the \textbf{\textit{wig}} package to demonstrate how to carry out the
computations in practice.
\cref{appendix} lists all the mathematical notations used throughout this note.

This note can be seen as a review of some selected topics on Computational Optimal Transport
and a relatively complete reference to understand the Wasserstein Index Generation model.
Moreover, this is a companion paper for the R package \textbf{\textit{wig}}\footnote{
  \textcolor{red}{TODO: CRAN url here}: \url{}
} to illustrate the usage of the package and how to solve the Optimal Transport problems as discussed in
\crefrange{sec:review}{sec:wig-model}.


\begin{figure}%[H]
  \centering
  \begin{tikzpicture}[
      round/.style={circle, draw=black, very thick, minimum size=8mm},
    ]\centering
    % nodes
    \node[round] (review)                            {2};
    \node[round] (sinkhorn)    [below=of review]     {3};
    \node[round] (sinkgrad)    [right=of sinkhorn]   {4};
    \node[round] (barycenter)  [below=of sinkhorn]   {5};
    \node[round] (barygrad)    [right=of barycenter] {6};
    \node[round] (wdl)         [below=of barygrad]   {7};
    \node[round] (wig)         [below=of wdl]        {8};
    \node[round] (wigr)        [below=of wig]        {9};

    % draw lines
    \draw[->,very thick] (review.south) -- (sinkhorn.north);
    \draw[->,very thick] (sinkhorn.east) -- (sinkgrad.west);
    \draw[->,very thick] (sinkhorn.south) -- (barycenter.north);
    \draw[->,very thick] (barycenter.east) -- (barygrad.west);
    \draw[->,very thick] (barygrad.south) -- (wdl.north);
    \draw[->,very thick] (wdl.south) -- (wig.north);
    \draw[->,very thick] (wig.south) -- (wigr.north);
  \end{tikzpicture}
  \caption{Diagram of the structure of the paper.}\label{fig:article-diagram}
\end{figure}


\cref{fig:article-diagram} provides a diagram of the structure of the paper.
For readers interested in entropic regularized OT problems,
they can refer to \cref{sec:review,sec:sinkhorn-algorithm} with optional \cref{sec:sinkhorn-gradient};
for readers interested in Wasserstein Barycenter problems,
they can refer to \cref{sec:review,sec:sinkhorn-algorithm,sec:wasserstein-barycenter}
with optional \cref{sec:barycenter-jacobian};
for readers interested in Wasserstein Dictionary Learning,
they can refer to \cref{sec:review,sec:sinkhorn-algorithm,sec:wasserstein-barycenter,sec:barycenter-jacobian,sec:wdl};
for readers interested in Wasserstein Index Generation model,
they can refer to everything from the WDL sections, plus \cref{sec:wig-model},
namely \cref{sec:review,sec:sinkhorn-algorithm,sec:wasserstein-barycenter,sec:barycenter-jacobian,sec:wdl,sec:wig-model};
\cref{sec:wig-package} is required if one needs to use the \textbf{\textit{wig}} package to carry out
any of the aforementioned computation.
