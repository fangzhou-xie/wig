

The Monge-Kantorovich problem\footnote{
  This is also called Kantorovich relaxation of the original Monge problem,
  as the problem was first proposed by \citet{monge1781} and then relaxed by \citet{kantorovich1942}.
  For mathematical foundation of Optimal Transport theory, the classic references are \citet{villani2003,villani2008};
  for its application in Economics, see \citet{galichon2016};
  for an introduction for applied mathematicians, see \citet{santambrogio2015a};
  for the computational aspects (algorithms and their properties) and its applications in data science and machine learning,
  see \citet{peyre2019}.
} states that:
given two probability vectors\footnote{
  Sometimes also called ``histograms'' in the Computational Optimal Transport community, for example in \citet{peyre2019}.
} $\mathbf{a} \in \Sigma_M$ and $\mathbf{b} \in \Sigma_N$,
how to find a \textit{coupling matrix} $\mathbf{P} \in \mathbb{R}_+^{M \times N}$
where each $\mathbf{P}_{ij}$ describes the flow of masses from bin $i$ to bin $j$,
such that the total cost of moving masses are optimal (minimal).
The cost of moving one unit of mass from bin $i$ to bin $j$ is $\mathbf{C}_{ij}$ and $\mathbf{C} \in \mathbb{R}_+^{M\times N}$.

\begin{problem}[Monge-Kantorovich]
Let $\mathbf{a} \in \Sigma_M$ and $\mathbf{b} \in \Sigma_N$.
The set of all coupling matrices is:

\begin{equation}\label{eqn-def:coupling-matrix}
  \begin{aligned}
    \mathbf{U(a,b)} \equiv \left\{
    \mathbf{P} \in \mathbb{R}_+^{M \times N}:
    \mathbf{P} \mathbb{1}_M = \mathbf{a}
    \text{ and }
    \mathbf{P}^\top \mathbb{1}_M = \mathbf{b}
    \right\}.
  \end{aligned}
\end{equation}
Given a cost matrix $\mathbf{C} \in \mathbb{R}^{M \times N}$,
the Kantorovich optimal transport problem tries to find the solution of the following

\begin{equation}\label{eqn:loss-kantorovich}
  \begin{aligned}
    L_{\mathbf{C}}(\mathbf{a}, \mathbf{b})
    \equiv \min_{\mathbf{P} \in \mathbf{U}(\mathbf{a},\mathbf{b})}
    \langle \mathbf{C}, \mathbf{P}\rangle
    = \sum_{i,j} \mathbf{C}_{ij} \mathbf{P}_{ij}.
  \end{aligned}
\end{equation}
\end{problem}

To solve this problem in practice, one needs to resort to linear programming\footnote{
  See \citet[Chapter 3]{peyre2019} for a historical overview and related algorithms.
}, which can be very challenging when the problem becomes large enough.
Instead of finding the exact solution to the above problem, one can actually add an entropy regularization term
to make this problem convex and hence greatly ease the computation.
This was first proposed in the seminal work of \citet{cuturi2013}
which leverages Sinkhorn-Knopp double scaling algorithm \citep{sinkhorn1964,sinkhorn1967,knight2008}.
\footnote{
  Therefore, in Optimal Transport literature,
  people usually refer to the numeric algorithm to solve the entropic regularized Kantorovich problem
  as the ``Sinkhorn algorithm''.
  % people sometimes refer to the entropic regularized Kantorovich problem
  % as the ``Sinkhorn problem'' and the algorithm to find its numeric solution as ``Sinkhorn algorithm.''
  Hence, its extensions are also named as ``X-khorn algorithms'',
  for example ``Greenkhorn algorithm'' \citep{altschuler2017} or ``Screenkhorn'' \citep{alaya2019}.
  Because of its simplicity, it has been re-discovered multiple times in history and renamed accordingly,
  for example,
  it can be linked to Sch\"odinger bridge problem \citep{schrodinger1931} or the RAS method \citep{bacharach1970}.
  See \citet{knight2008,leonard2013,modin2024} for historical accounts.
}

\begin{problem}[{\cite[Entropic Regularized OT Problem]{cuturi2013,peyre2019}}]\label{thm:entropic-regularized-OT-problem}
Given the coupling matrix $\mathbf{P} \in \mathbb{R}_+^{M \times N}$, its discrete entropy is defined as
\begin{equation*}
  \begin{aligned}
    \mathbf{H}(\mathbf{P})
    \equiv - \langle \mathbf{P}, \log(\mathbf{P}) - 1\rangle
    = - \sum_{i,j} \mathbf{P}_{ij} (\log(\mathbf{P}_{ij}) - 1).
  \end{aligned}
\end{equation*}

With the cost matrix $\mathbf{C}$, the entropic regularized loss function is

\begin{equation}\label{eqn:entropic-regularized-OT-loss}
  \begin{aligned}
    % \min_{\mathbf{P} \in \mathbf{U}(\mathbf{a},\mathbf{b})}
    L^\varepsilon_{\mathbf{C}}(\mathbf{a}, \mathbf{b})
    \equiv
    % \min_{\mathbf{P} \in \mathbf{U}(\mathbf{a},\mathbf{b})}
    \langle \mathbf{P}, \mathbf{C}\rangle - \varepsilon \mathbf{H}(\mathbf{P}),
  \end{aligned}
\end{equation}

and the entropic regularized OT problem is thus

\begin{equation}\label{eqn:entropic-regularized-OT-problem}
  \begin{aligned}
    \min_{\mathbf{P} \in \mathbf{U}(\mathbf{a},\mathbf{b})}
    L^\varepsilon_{\mathbf{C}}(\mathbf{a}, \mathbf{b})
    \equiv
    \min_{\mathbf{P} \in \mathbf{U}(\mathbf{a},\mathbf{b})}
    \langle \mathbf{P}, \mathbf{C}\rangle - \varepsilon \mathbf{H}(\mathbf{P}).
  \end{aligned}
\end{equation}
\end{problem}
% \footnotetext{
%   Naturally, this loss function is sometimes referred to as ``Sinkhorn loss.''
% }

The Sinkhorn problem has a unique optimal solution, since it is $\varepsilon$-strongly convex \citep{peyre2019}.
% Let us denote the solution to \cref{eqn:sinkhorn-problem} as $\mathbf{P}_\varepsilon$
% The Sinkhorn problem converges to original Kantorovich problem as $\varepsilon \to 0$ \citep[Proposition 4.1]{peyre2020}.
% And one can show that the solution of \cref{thm:problem-sinkhorn} 

\begin{proposition}[{\cite[Proposition 4.3]{peyre2019}}]
  The solution to \cref{eqn:entropic-regularized-OT-problem} is unique and has the form
  \begin{equation}\label{eqn:regularized-OT-solution}
    \begin{aligned}
      \mathbf{P}_{ij} = \mathbf{u}_i \mathbf{K}_{ij} \mathbf{v}_j,
    \end{aligned}
  \end{equation}
  or in matrix notation
  \begin{equation}\label{eqn:regularized-OT-solution-matrix-form}
    \begin{aligned}
      \mathbf{P} = \diag(\mathbf{u}) \,\mathbf{K} \,\diag(\mathbf{v})
    \end{aligned}
  \end{equation}
  where $i \in \left\{1, \ldots, M\right\}$ and $j \in \left\{1, \ldots, N\right\}$ with two (unknown)
  scaling variables $\mathbf{u} \in \mathbb{R}^M$ and $\mathbf{v} \in \mathbb{R}^N$.
\end{proposition}

\begin{proof}
  Let $\mathbf{f} \in \mathbb{R}^M$ and $\mathbf{g} \in \mathbb{R}^N$ be two dual variables for the two constraints
  $\mathbf{P} \mathbb{1}_N = \mathbf{a}$ and $\mathbf{P}^\top \mathbb{1}_M = \mathbf{b}$, respectively.
  The Lagrangian of \cref{eqn:entropic-regularized-OT-problem} becomes
  \begin{equation}\label{eqn:regularized-OT-lagrangian}
    \begin{aligned}
      \mathscr{L}(\mathbf{P}, \mathbf{f}, \mathbf{g}) =
      \langle \mathbf{P}, \mathbf{C}\rangle - \varepsilon \mathbf{H}(\mathbf{P}) -
      \langle \mathbf{f}, \mathbf{P} \mathbb{1}_m - \mathbf{a}\rangle  -
      \langle \mathbf{g}, \mathbf{P}^\top \mathbb{1}_n - \mathbf{b}\rangle.
    \end{aligned}
  \end{equation}
  The first-order-condition gives us
  \begin{equation*}
    \begin{aligned}
      \frac{
        \partial \mathscr{L}(\mathbf{P}, \mathbf{f}, \mathbf{g})
      }{
        \partial \mathbf{P}_{ij}
      } = \mathbf{C}_{ij} + \varepsilon \log(\mathbf{P}_{ij}) - \mathbf{f}_i - \mathbf{g}_j = 0
    \end{aligned}
  \end{equation*}
  Therefore we can solve for the optimal coupling matrix $\mathbf{P}$ as:
  \begin{equation*}
    \begin{aligned}
      \log(\mathbf{P}_{ij})
      =
      \frac{\mathbf{f}_i}{\varepsilon}
      \left(-\frac{\mathbf{C}_{ij}}{\varepsilon}\right)
      \frac{\mathbf{g}_j}{\varepsilon}
    \end{aligned}
  \end{equation*}
  or

  \begin{equation}\label{eqn:optimal-coupling}
    \begin{aligned}
      \mathbf{P}_{ij}
      =
      \exp(\frac{\mathbf{f}_i}{\varepsilon})
      \exp(-\frac{\mathbf{C}_{ij}}{\varepsilon})
      \exp(\frac{\mathbf{g}_j}{\varepsilon})
      =
      \mathbf{u}_i \mathbf{C}_{ij} \mathbf{v}_j,
    \end{aligned}
  \end{equation}

  where $\mathbf{u}_i = \exp(\frac{\mathbf{f}_i}{\varepsilon})$,
  $\mathbf{v}_j = \exp(\frac{\mathbf{g}_j}{\varepsilon})$, and
  $\mathbf{K}_{ij} = \exp(-\frac{\mathbf{C}_{ij}}{\varepsilon})$, respectively.
\end{proof}

\begin{remark}[]
  Note that computation of $\mathbf{P}_{ij}$ are effectively in the exponential domain, if we choose to find
  optimal solution for $\mathbf{P}$ by updating $\mathbf{f}$ and $\mathbf{g}$.
  This is indeed the case for the vanilla Sinkhorn Algorithm, and hence it has numeric instability issue
  which need to be overcome by the ``log-stabilization'' technique,
  which moves all computation in the log-domain (\cref{subsec:log-sinkhorn}).
\end{remark}
