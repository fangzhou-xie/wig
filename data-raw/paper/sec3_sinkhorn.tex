
In this section, I will discuss three different versions of Sinkhorn algorithms.

\subsection{Vanilla Sinkhorn Algorithm}\label{subsec:vanilla-sinkhorn}

Given the optimal solution to the Sinkhorn problem in \cref{eqn:regularized-OT-solution-matrix-form},
with the constraints in \cref{eqn-def:coupling-matrix}, we have

\begin{equation}
  \begin{aligned}
    \diag(\mathbf{u}) \,  \mathbf{K}      \, \diag(\mathbf{v}) \mathbb{1}_N  = \mathbf{a}
    \quad\text{and}\quad
    \diag(\mathbf{v}) \,  \mathbf{K}^\top \, \diag(\mathbf{u}) \mathbb{1}_M  = \mathbf{b}.
  \end{aligned}
\end{equation}

Since we also have $\diag(\mathbf{v}) \mathbb{1}_N = \mathbf{v}$ and $\diag(\mathbf{u}) \mathbb{1}_M = \mathbf{u}$,
we have that

\begin{equation}\label{eqn:vanilla-sinkhorn-solution-multiply}
  \begin{aligned}
    \mathbf{u} \odot (\mathbf{K} \mathbf{v}) = \mathbf{a}
    \quad\text{and}\quad
    \mathbf{v} \odot (\mathbf{K}^\top \mathbf{u}) = \mathbf{b}.
  \end{aligned}
\end{equation}

Equivalently, we can also write the solution as

\begin{equation}\label{eqn:vanilla-sinkhorn-solution-division}
  \begin{aligned}
    \mathbf{u} = \mathbf{a} \oslash (\mathbf{K} \mathbf{v})
    \quad\text{and}\quad
    \mathbf{v} = \mathbf{b} \oslash (\mathbf{K}^\top \mathbf{u}).
  \end{aligned}
\end{equation}

In the above equations, $\odot$ and $\oslash$ refer to element-wise multiplication and division, respectively.
Therefore, we can leverage \cref{eqn:vanilla-sinkhorn-solution-division} to update $\mathbf{u}$ and $\mathbf{v}$
till convergence and conclude with optimal coupling $\mathbf{P}$ by \cref{eqn:regularized-OT-solution-matrix-form}.
% Therefore, at any iteration $l$, we have the Sinkhorn algorithm updates with arbitrary $\mathbf{v}^{(0)}$.


\begin{update}[Vanilla Sinkhorn Updates]\label{update:vanilla-sinkhorn}
  At any iteration $\ell = 1, \ldots, L$
  we can update the scaling variables $\mathbf{u}$ and $\mathbf{v}$ by the following equations:
  \begin{equation}
    \begin{aligned}
      \mathbf{u}^{(\ell)} \equiv \mathbf{a} \oslash (\mathbf{K}\mathbf{v}^{(\ell-1)})
      \quad\text{and}\quad
      \mathbf{v}^{(\ell)} \equiv \mathbf{b} \oslash (\mathbf{K}^\top \mathbf{u}^{(\ell)}).
    \end{aligned}
  \end{equation}
\end{update}

\begin{remark}
  In \cref{update:vanilla-sinkhorn}, the choice of $\mathbf{v}^{(0)}$ is rather arbitrary,
  as long as it is not $\mathbb{0}$.
  The most common choice is to consider the vectors of ones, i.e. $\mathbb{1}_N$.
  Changing the initialization vector will result in different solutions to $\mathbf{u}$ and $\mathbf{v}$,
  since they are only defined up to a multiplicative constant.
  But $\mathbf{u}$ and $\mathbf{v}$ will converge regardless of the initialization,
  and result in the same optimal coupling matrix $\mathbf{P}$ upon convergence.
  % But the optimal coupling will result in the same matrix (once reaching convergence),
  % regardless of the initialization of $\mathbf{v}$.
  See \citet[p.82]{peyre2019} for discussion.
\end{remark}

After $L$ iterations, the scaling variables $\mathbf{u}^{(L)}$ and $\mathbf{v}^{(L)}$ can be used to compute the
optimal coupling matrix $\mathbf{P} = \diag\mathbf{u}^{(L)} \mathbf{K} \diag \mathbf{v}^{(L)}$.

From \cref{update:vanilla-sinkhorn}, we can arrive at the vanilla Sinkhorn algorithm (\cref{algo:vanilla-sinkhorn}).

\begin{algorithm}[H]
  \caption{Vanilla Sinkhorn}
  \begin{algorithmic}[1]\label{algo:vanilla-sinkhorn}
    \Require $\mathbf{a} \in \Sigma_M$, $\mathbf{b} \in \Sigma_N$, $\mathbf{C} \in \mathbb{R}^{M\times N}$, $\varepsilon > 0$.
    \Initialize $\mathbf{u} = \mathbb{1}_M$, $\mathbf{v} = \mathbb{1}_N$.
    \State $\mathbf{K} = \exp(-\frac{\mathbf{C}}{\varepsilon})$
    \While{Not Convergence}
    \State $\mathbf{u} \leftarrow \mathbf{a} \oslash (\mathbf{K} \mathbf{v})$
    \State $\mathbf{v} \leftarrow \mathbf{b} \oslash (\mathbf{K}^\top \mathbf{u})$
    \EndWhile
    \State $\mathbf{P} = \diag(\mathbf{u}) \, \mathbf{K} \, \diag(\mathbf{v})$
    \Ensure Optimal Coupling $\mathbf{P}$
  \end{algorithmic}
\end{algorithm}

\begin{remark}[Termination/Convergence Condition]
  Since we have \cref{eqn:vanilla-sinkhorn-solution-multiply}, we can use those equations to check if the scaling variables
  $\mathbf{u}$ and $\mathbf{v}$ at the current iteration could approximate $\mathbf{a}$ and $\mathbf{b}$ well.
  That is, for some small $\rho > 0$,
  \begin{equation*}
    \begin{aligned}
      \lVert \mathbf{u}^{(\ell)} \odot \left(\mathbf{K} \mathbf{v}^{(\ell)}\right) - \mathbf{a}\rVert_2 \le \rho
      \quad\text{and}\quad
      \lVert \mathbf{v}^{(\ell)} \odot \left(\mathbf{K}^\top \mathbf{u}^{(\ell)}\right) - \mathbf{b}\rVert_2 \le \rho
    \end{aligned}
  \end{equation*}
  If the norms are smaller than some threshold $\rho$, we can then terminate the program.
\end{remark}


\subsection{Parallel Sinkhorn Algorithm}\label{subsec:parallel-sinkhorn}

Note that in the Vanilla Sinkhorn Algorithm, the computation are carried out by vector-vector element-wise computation,
i.e., $\mathbf{u}, \mathbf{a}, \text{and } \mathbf{K}\mathbf{v} \in \mathbb{R}^M$
and $\mathbf{v}, \mathbf{b}, \text{and } \mathbf{K}^\top \mathbf{u} \in \mathbb{R}^N$.
That is, if we have a single source $\mathbf{a}$ and a single target $\mathbf{b}$,
we are computing the ``one-to-one'' transportation.
However, sometimes we need to calculate one-to-many, many-to-one, or many-to-many transport
% \footnote{
%   In fact, the parallelization described here can also be ``many-to-many'', as long as each source only correspond to
%   one target, and the total number of sources match with that of the targets.
%   This is the case in \citet[Remark 4.16]{peyre2020}.
%   Here, I only consider the case of having one source, but with arbitrarily many targets (or vice-versa),
%   as this would become useful for the later discussion on Wasserstein dictionary learning.
% },
and this can be easily computed in a parallelized fashion.
All we need to do is to define the matrices $\mathbf{A}$ and $\mathbf{B}$ instead of the vectors $\mathbf{a}$ and $\mathbf{b}$,
and all that remains is to carry out the matrix computation instead of the vector ones.

\begin{definition}
  For some $S \in \mathbb{N}_+$:
  \begin{itemize}
    \item Suppose we have $\mathbf{a} \in \Sigma_M$ and $\mathbf{b}_1, \ldots, \mathbf{b}_S \in \Sigma_N$.
          Then we can define
          \begin{equation*}
            \begin{aligned}
              \mathbf{A} = \left[\mathbf{a}, \ldots, \mathbf{a}\right] \in \mathbb{R}^{M \times S}
              \quad\text{and}\quad
              \mathbf{B} = \left[\mathbf{b}_1, \ldots, \mathbf{b}_S \right] \in \mathbb{R}^{N \times S}.
            \end{aligned}
          \end{equation*}
    \item Suppose we have $\mathbf{a}_1, \ldots, \mathbf{a}_S \in \Sigma_M$ and $\mathbf{b} \in \Sigma_N$.
          Then we can define
          \begin{equation*}
            \begin{aligned}
              \mathbf{A} = \left[\mathbf{a}_1, \ldots, \mathbf{a}_S\right] \in \mathbb{R}^{M \times S}
              \quad\text{and}\quad
              \mathbf{B} = \left[\mathbf{b}, \ldots, \mathbf{b} \right] \in \mathbb{R}^{N \times S}.
            \end{aligned}
          \end{equation*}
    \item Suppose we have $\mathbf{a}_1, \ldots, \mathbf{a}_S \in \Sigma_M$ and $\mathbf{b}_1, \ldots, \mathbf{b}_S \in \Sigma_N$.
          Then we can define
          \begin{equation*}
            \begin{aligned}
              \mathbf{A} = \left[\mathbf{a}_1, \ldots, \mathbf{a}_S\right] \in \mathbb{R}^{M \times S}
              \quad\text{and}\quad
              \mathbf{B} = \left[\mathbf{b}_1, \ldots, \mathbf{b}_S \right] \in \mathbb{R}^{N \times S}.
            \end{aligned}
          \end{equation*}
  \end{itemize}
\end{definition}

\begin{remark}[]
  In the many-to-many case, the number of columns in $\mathbf{A}$ should be equal to that of $\mathbf{B}$;
  % This effectively means we are calculating the scaling variable $\mathbf{u}_d$ and $\mathbf{v}_d$ 
  whereas in one-to-many or many-to-one cases, we need to duplicate the single vector $S$ times,
  as if we had a $S$-column matrix to begin with.
  In many modern linear algebra packages, this can be achieved automatically by ``broadcasting,''\footnote{
    For example,
    Eigen (C++): \url{https://eigen.tuxfamily.org/dox/group__TutorialReductionsVisitorsBroadcasting.html},
    NumPy (Python): \url{https://numpy.org/doc/stable/user/basics.broadcasting.html}.
  }
  without the need of actually doing the duplication of the column vectors.
  Or in terms of matrix computation,
  for example,
  $\mathbf{B} = \left[\mathbf{b}, \ldots, \mathbf{b}\right] = \mathbb{1}_S^\top \otimes \mathbf{b}$.
\end{remark}

Therefore, we can have the matrix version\footnote{
  Instead of computing the vector-vector computation in \cref{eqn:vanilla-sinkhorn-solution-division} $S$ times,
  we are computing a matrix-matrix product/division once.
  They are mathematically equivalent, but in practice,
  the latter would be much faster due to the implementation of linear algebra routines (e.g. BLAS).
  Hence, this simple extension is considered parallel since we process $S$ columns of data in one batch.
} of \cref{eqn:vanilla-sinkhorn-solution-division} with scaling variable
$\mathbf{U} \in \mathbb{R}^{M\times S}$ and $\mathbf{V} \in \mathbb{R}^{N\times S}$,

\begin{equation}
  \begin{aligned}
    \mathbf{U} = \mathbf{A} \oslash (\mathbf{K}\mathbf{V})
    \quad\text{and}\quad
    \mathbf{V} = \mathbf{B} \oslash (\mathbf{K}^\top \mathbf{V}).
  \end{aligned}
\end{equation}

Hence we could have the matrix version of the updating equations of Sinkhorn algorithm.

\begin{update}\label{update:parallel-sinkhorn}
  At any iteration $\ell = 1, \ldots, L$ and initialized $\mathbf{V}^{(0)} \in \mathbb{R}^{N\times S}$, we have
  \begin{equation}
    \begin{aligned}
      \mathbf{U}^{(\ell)} \equiv \mathbf{A} \oslash (\mathbf{K} \mathbf{V}^{(\ell-1)})
      \quad\text{and}\quad
      \mathbf{V}^{(\ell)} \equiv \mathbf{B} \oslash (\mathbf{K}^\top \mathbf{U}^{(\ell)}).
    \end{aligned}
  \end{equation}
\end{update}

\begin{remark}[]
  Since \cref{update:parallel-sinkhorn} only involves matrix operations,
  the Sinkhorn algorithm are intrinsically GPU friendly, as matrix operations are even more efficient on GPU than on CPU.
\end{remark}


\begin{algorithm}[H]
  \caption{Parallel Sinkhorn Algorithm}
  \begin{algorithmic}[1]\label{algo:parallel-sinkhorn}
    \Require $\mathbf{A} \in \Sigma_{M\times S}$, $\mathbf{B}\in \Sigma_{N\times S}$, $\mathbf{C} \in \mathbb{R}^{M\times N}$, $\varepsilon > 0$.
    \Initialize $\mathbf{U} = \mathbb{1}_{M \times S}$, $\mathbf{V} = \mathbb{1}_{N \times S}$.
    \State $\mathbf{K} = \exp(-\frac{\mathbf{C}}{\varepsilon})$
    \While{Not Convergence}
    \State $\mathbf{U} \leftarrow \mathbf{A} \oslash (\mathbf{K} \mathbf{V})$
    \State $\mathbf{V} \leftarrow \mathbf{B} \oslash (\mathbf{K}^\top \mathbf{U})$
    \EndWhile
    % \State $\mathbf{P} = diag(\mathbf{u}) \, \mathbf{K} \, diag(\mathbf{v})$
    \State $\forall s, \mathbf{P}_s = diag(\mathbf{U}_s) \, \mathbf{K} \, diag(\mathbf{V}_s)$
    \Ensure Optimal Coupling $\mathbf{P}_s, \forall s$
  \end{algorithmic}
\end{algorithm}

\begin{remark}[Convergence Condition]
  Similar to the termination condition in \cref{subsec:vanilla-sinkhorn}, we now have the matrix norms to check convergence.
  For some $\rho > 0$,
  \begin{equation*}
    \begin{aligned}
      \lVert
      \mathbf{U}^{(\ell)} \odot \left(\mathbf{K} \mathbf{V}^{(\ell)}\right) - \mathbf{A}
      \rVert_2 \le \rho,
      \quad\text{and}\quad
      \lVert
      \mathbf{V}^{(\ell)} \odot \left(\mathbf{K}^\top \mathbf{U}^{(\ell)}\right) - \mathbf{B}
      \rVert_2 \le \rho.
    \end{aligned}
  \end{equation*}
\end{remark}


\subsection{Log-stabilized Sinkhorn}\label{subsec:log-sinkhorn}

As can be seen in \cref{eqn:optimal-coupling,eqn:vanilla-sinkhorn-solution-division},
the updates on $\mathbf{u}$ and $\mathbf{v}$ are in the exponential domain.
Therefore, if either $\mathbf{C}_{ij}$ is large or $\varepsilon$ is small,
the corresponding $\mathbf{K}_{ij}$ would be very large,
and the subsequent updates will fail because of the Inf/NaN values introduced by the numeric overflow.
This is commonly known as the numeric instability issue of the Sinkhorn algorithm \citep[Chapter 4.4]{peyre2019}.
% We will leverage the ``log-sum-exp'' trick\footnote{
%   \url{https://en.wikipedia.org/wiki/LogSumExp}
% } to stabilize the computation and arrive at the updating equations.

To circumvent this instability issue, we could move the computation into the log-domain,
therefore without the need for computation of the exponential function.
To this end, instead of updating the $\mathbf{u}$ and $\mathbf{v}$ as in \cref{update:vanilla-sinkhorn},
we can consider the dual variables $\mathbf{f}$ and $\mathbf{g}$ in the Lagrangian in \cref{eqn:regularized-OT-lagrangian}.
Since $\mathbf{u} = \exp \left(\frac{\mathbf{f}}{\varepsilon}\right)$
and $\mathbf{v} = \exp \left(\frac{\mathbf{g}}\varepsilon\right)$, we have

\begin{equation}\label{eqn:fg-expression-as-uv}
  \begin{aligned}
    \mathbf{f} = \varepsilon \log(\mathbf{u})
    \quad\text{and}\quad
    \mathbf{g} = \varepsilon \log(\mathbf{v}).
  \end{aligned}
\end{equation}

Then, we can also rewrite the elements in the coupling matrix $\mathbf{P}$ as

\begin{equation*}
  \begin{aligned}
    \mathbf{P}_{ij}^{(\ell)}
    = \mathbf{u}_i^{(\ell)} \mathbf{K}_{ij} \mathbf{v}_j^{(\ell)}
    = \exp \left(
    -\frac{\mathbf{C}_{ij} - \mathbf{f}^{(\ell)}_i - \mathbf{g}^{(\ell)}_j}{\varepsilon}
    \right)
  \end{aligned}
\end{equation*}

and the matrix form of $\mathbf{P}$ expressed in $\mathbf{f}$ and $\mathbf{g}$ is
% after $L$ iterations, is

\begin{equation}\label{eqn:optimal-coupling-fg}
  \begin{aligned}
    \mathbf{P}^{(\ell)} = \exp \left(
    - \frac{\mathbf{C} - \mathbf{f}^{(\ell)} \cdot \mathbb{1}_N^\top -
      \mathbb{1}_M \cdot \mathbf{g}^{(\ell)\top}}{\varepsilon}
    \right).
  \end{aligned}
\end{equation}

Let us also denote a function $\mathcal{P}(\mathbf{f}, \mathbf{g})$ with
$\mathbf{f} \in \mathbb{R}^M$ and $\mathbf{g} \in \mathbb{R}^N$,

\begin{equation}\label{eqn:function-P}
  \begin{aligned}
    \mathcal{P}(\mathbf{f}, \mathbf{g})
    = \exp \left(
    - \frac{
      \mathbf{C} - \mathbf{f} \cdot \mathbb{1}_N^\top - \mathbb{1}_M \cdot \mathbf{g}^\top
    }{\varepsilon}
    \right),
  \end{aligned}
\end{equation}

and rewrite $\mathbf{P}^{(\ell)} = \mathcal{P}\left(\mathbf{f}^{(\ell)}, \mathbf{g}^{(\ell)}\right)$.
Note that $\mathbf{f}$ and $\mathbf{g}$ are dual variables in the Lagrangian in \cref{eqn:regularized-OT-lagrangian}.
Then we can plug in $\mathbf{u}$ and $\mathbf{v}$ to get the following updating equations

\begin{equation}\label{eqn:log-sinkhorn-update1}
  \begin{aligned}
    \mathbf{f}^{(\ell)}
     & = \varepsilon\log \mathbf{u}^{(\ell)}
    = \varepsilon\log \mathbf{a} - \varepsilon\log (\mathbf{K} \mathbf{v}^{(\ell-1)}), \\
    \mathbf{g}^{(\ell)}
     & = \varepsilon\log \mathbf{v}^{(\ell)}
    = \varepsilon\log \mathbf{b} = \varepsilon\log (\mathbf{K}^\top \mathbf{u}^{(\ell)})
  \end{aligned}
\end{equation}

Since $\mathbf{K}\mathbf{v}^{(\ell-1)} \in \mathbb{R}^m$, $\forall i$,
the $i$-th element is %of $-\varepsilon\log (\mathbf{K} \mathbf{v})$ is

\begin{equation*}
  \begin{aligned}
    \left(\mathbf{K}\mathbf{v}^{(\ell-1)}\right)_i
     & = \sum_j \mathbf{K}_{ij} \mathbf{v}_j^{(\ell-1)}                                                                      \\
     & = \sum_j \exp(-\frac{\mathbf{C}_{ij}}{\varepsilon}) \exp(\frac{\mathbf{g}_j^{(\ell-1)}}{\varepsilon})                 \\
    %  & = \sum_j \exp(- \frac{\mathbf{C}_{ij} - \mathbf{g}_j}{\varepsilon})                                                        \\
    %  & = \sum_j \exp(- \frac{\mathbf{C}_{ij} - \mathbf{f}_i - \mathbf{g}_j}{\varepsilon})
    % \exp(-\frac{\mathbf{f}_i}{\varepsilon})                                                            \\
     & = \left[\sum_j \exp(- \frac{\mathbf{C}_{ij} - \mathbf{f}_i^{(\ell-1)} - \mathbf{g}_j^{(\ell-1)}}{\varepsilon})\right]
    \exp(-\frac{\mathbf{f}_i^{(\ell-1)}}{\varepsilon}),
  \end{aligned}
\end{equation*}

and

\begin{equation*}
  \begin{aligned}
    -\varepsilon\log \left(\mathbf{K} \mathbf{v}^{(\ell-1)}\right)_i
     & =
    -\varepsilon\log \sum_j \exp(- \frac{\mathbf{C}_{ij} - \mathbf{f}_i^{(\ell-1)} - \mathbf{g}_j^{(\ell-1)}}{\varepsilon})
    + \mathbf{f}_i^{(\ell-1)} \\
     & =
    -\varepsilon\log \left[
      \exp \left(
      -\frac{\mathbf{C} - \mathbf{f}^{(\ell-1)} \mathbb{1}_N^\top - \mathbb{1}_M (\mathbf{g}^{(\ell-1)})^\top}{\varepsilon}
      \right)
      \cdot \mathbb{1}_N
      \right]_i
    + \mathbf{f}_i^{(\ell-1)} \\
     & =
    -\varepsilon\log
    \left[
      \mathcal{P} \left(\mathbf{f}^{(\ell-1)}, \mathbf{g}^{(\ell-1)}\right)\cdot \mathbb{1}_N
      \right]_i + \mathbf{f}_i^{(\ell-1)}.
  \end{aligned}
\end{equation*}

Therefore, we have

\begin{equation}\label{eqn:neg-log-kv-vec}
  \begin{aligned}
    - \varepsilon \log \left(\mathbf{K} \mathbf{v}^{(\ell)}\right)
    = - \varepsilon \log \left[
      \mathcal{P} \left(\mathbf{f}^{(\ell)}, \mathbf{g}^{(\ell)}\right)\cdot \mathbb{1}_N
      \right] + \mathbf{f}^{(\ell)}.
  \end{aligned}
\end{equation}

And we have the updating equation for $\mathbf{f}^{(\ell)}$.
If we follow similar steps, we can also get it for $\mathbf{g}^{(\ell)}$.
That is,

\begin{equation}\label{eqn:update-fg-by-P}
  \begin{aligned}
    \mathbf{f}^{(\ell)}
     & = \mathbf{f}^{(\ell-1)} + \varepsilon\log \mathbf{a} -\varepsilon\log
    \left[
      \mathcal{P} \left(\mathbf{f}^{(\ell-1)}, \mathbf{g}^{(\ell-1)}\right)\cdot \mathbb{1}_N
    \right],                                                                 \\
    \mathbf{g}^{(\ell)}
     & = \mathbf{g}^{(\ell-1)} + \varepsilon\log \mathbf{b} -\varepsilon\log
    \left[
      \mathcal{P}^\top \left(\mathbf{f}^{(\ell)}, \mathbf{g}^{(\ell-1)}\right) \cdot \mathbb{1}_M
      \right].
  \end{aligned}
\end{equation}


Note that $\log \left[\mathcal{P}(\mathbf{f}, \mathbf{g})\cdot \mathbb{1}_N\right]$
is essentially a ``log-sum-exp'' form, and we can apply the ``log-sum-exp'' trick\footnote{
  \url{https://en.wikipedia.org/wiki/LogSumExp}
} to make this equation numeric stable while computing it in practice.
Therefore, let us define the following two functions,

\begin{equation}\label{eqn:function-c-and-Q}
  \begin{aligned}
    \mathcal{Q} \left(\mathbf{f}, \mathbf{g}\right)
    % = \mathcal{Q} \left(\mathbf{f}, \mathbf{g}, c\right)
     & = \exp \left(
    - \frac{
      \mathbf{C} - \mathbf{f} \cdot \mathbb{1}_N^\top - \mathbb{1}_M \cdot \mathbf{g}^\top
      - c \left(\mathbf{f}, \mathbf{g}\right)
    }{\varepsilon}
    \right), \text{ where }                                                                                        \\
    c \left(\mathbf{f}, \mathbf{g}\right)
     & = \min \left\{\mathbf{C} - \mathbf{f} \cdot \mathbb{1}_N^\top - \mathbb{1}_M \cdot \mathbf{g}^\top\right\},
  \end{aligned}
\end{equation}


we can again rewrite \cref{eqn:update-fg-by-P} as

\begin{dgroup*}\label{eqn:update-fg-by-Q}
  \begin{dmath*}
    \mathbf{f}^{(\ell)}
    = \mathbf{f}^{(\ell-1)} + \varepsilon\log \mathbf{a}
    -\varepsilon\log \left[
      \mathcal{P} \left(\mathbf{f}^{(\ell-1)}, \mathbf{g}^{(\ell-1)}\right)\cdot \mathbb{1}_N
      \right],                                                                                                              \\
    = \mathbf{f}^{(\ell-1)} + \varepsilon\log \mathbf{a}
    -\varepsilon\log \left[
      \exp \left(
      -\frac{
        \mathbf{C} - \mathbf{f}^{(\ell-1)} \cdot \mathbb{1}_N^\top - \mathbb{1}_M \cdot \mathbf{g}^{(\ell-1)\top}
        % - c \left(\mathbf{f}^{(\ell-1)}, \mathbf{g}^{(\ell-1)}\right)
      }{\varepsilon}
      \right)\cdot \mathbb{1}_N
      \right]
    = \mathbf{f}^{(\ell-1)} + \varepsilon\log \mathbf{a}
    -\varepsilon\log \left[
      \exp \left(
      -\frac{c \left(\mathbf{f}^{(\ell-1)}, \mathbf{g}^{(\ell-1)}\right)}\varepsilon
      \right)
      \mathcal{Q}\left(\mathbf{f}^{(\ell-1)}, \mathbf{g}^{(\ell-1)}\right) \cdot \mathbb{1}_N
      \right]
    \\
    = \mathbf{f}^{(\ell-1)} + \varepsilon\log \mathbf{a} + c \left(\mathbf{f}^{(\ell-1)}, \mathbf{g}^{(\ell-1)}\right)
    -\varepsilon\log
    \left[
      \mathcal{Q} \left(
      \mathbf{f}^{(\ell-1)}, \mathbf{g}^{(\ell-1)}
      \right)\cdot \mathbb{1}_N
      \right],
  \end{dmath*}
  \begin{dmath*}
    \mathbf{g}^{(\ell)}
    = \mathbf{g}^{(\ell-1)} + \varepsilon\log \mathbf{b} + c \left(\mathbf{f}^{(\ell)}, \mathbf{g}^{(\ell-1)}\right)
    -\varepsilon\log
    \left[
      \mathcal{Q}^\top \left(\mathbf{f}^{(\ell)}, \mathbf{g}^{(\ell-1)}\right) \cdot \mathbb{1}_M
      \right].
  \end{dmath*}
\end{dgroup*}



Thus, we have obtained the numerically stable version of the updating equations for $\mathbf{f}$ and $\mathbf{g}$.

% Here, we subtract elements from both $\mathbf{f}^{(\ell-1)}$ and $\mathbf{g}^{(\ell-1)}$ from $\mathbf{C}_{ij}$,
% and the reason would become apparent once we derive the optimal coupling solution in terms of dual variables in
% \cref{eqn:optimal-coupling-fg}.

% With similar steps, we can obtain the expression for $\mathbf{g}^{(\ell)}$,
% and then we can rewrite the updating equations in \cref{eqn:log-sinkhorn-update1} as the following.

\begin{update}\label{update:log-sinkhorn}
  At any iteration $\ell = 1, \ldots, L$ and initialized $\mathbf{f}^{(0)}$ and $\mathbf{g}^{(0)}$, we have
  \begin{equation}\label{eqn:log-sinkhorn-update}
    \begin{aligned}
      \mathbf{f}^{(\ell)}
       & = \mathbf{f}^{(\ell-1)} + \varepsilon\log \mathbf{a} + c \left(\mathbf{f}^{(\ell-1)}, \mathbf{g}^{(\ell-1)}\right)
      -\varepsilon\log
      \left[
        \mathcal{Q} \left(
        \mathbf{f}^{(\ell-1)}, \mathbf{g}^{(\ell-1)}
        \right)\cdot \mathbb{1}_N
      \right],                                                                                                              \\
      \mathbf{g}^{(\ell)}
       & = \mathbf{g}^{(\ell-1)} + \varepsilon\log \mathbf{b} + c \left(\mathbf{f}^{(\ell)}, \mathbf{g}^{(\ell-1)}\right)
      -\varepsilon\log
      \left[
        \mathcal{Q}^\top \left(\mathbf{f}^{(\ell)}, \mathbf{g}^{(\ell-1)}\right) \cdot \mathbb{1}_M
        \right].
    \end{aligned}
  \end{equation}
\end{update}

\begin{remark}[]
  As mentioned in \cref{subsec:vanilla-sinkhorn}, the initialization for $\mathbf{u}$ and $\mathbf{v}$ are vectors of ones.
  This is to say that, the initializations of $\mathbf{f}$ and $\mathbf{g}$ are vectors of zeros,
  i.e. $\mathbf{f}^{(0)} = \mathbb{0}_M$ and $\mathbf{g}^{(0)} = \mathbb{0}_N$.
\end{remark}


% Note that for any matrix $\mathbf{M} \in \mathbb{R}^{M \times N}$ and $\varepsilon > 0$,

% \begin{equation}\label{eqn:log-sum-exp-matrix-form}
%   \begin{aligned}
%     - \varepsilon \log \left[
%       \exp \left(-\frac{\mathbf{M}}{\varepsilon}\right) \mathbb{1}_N
%       \right]
%     = \underbar{M} - \varepsilon\log \left[
%       \exp \left(-\frac{\mathbf{M} - \underbar{M}}{\varepsilon}\right) \mathbb{1}_N
%       \right].
%   \end{aligned}
% \end{equation}

% This is essentially the ``log-sum-exp'' trick\footnote{
%   \url{https://en.wikipedia.org/wiki/LogSumExp}
% } to stabilize the computation.
% Thus, \cref{eqn:log-sinkhorn-update,eqn:log-sinkhorn-update-stable} are equivalent,
% but the latter one is numeric stable as it is always bounded.
% Therefore,
% For ease of notation, however,
% I will use the form in \cref{eqn:log-sinkhorn-update} in the remainder of this note;
% in practice, it will always be implemented as in \cref{eqn:log-sinkhorn-update-stable}.


% and should be implemented as such for \cref{eqn:log-sinkhorn-update} in practice.
% For simplicity of expression, however, we will keep the form as in \cref{update:log-sinkhorn-update} to simplify notation.

% softmin function
% \begin{definition}[Softmin function]
%   Given $\varepsilon > 0$ and any vector $\mathbf{z}$, the Softmin function
%   \begin{equation}
%     \begin{aligned}
%       \min_{\varepsilon} \mathbf{z}
%        & = - \varepsilon\log \sum_i \exp(-\frac{\mathbf{z}_i}{\varepsilon})
%       = \ubar{z} - \varepsilon\log \sum_i \exp(-\frac{\mathbf{z}_i-\ubar{z}}{\varepsilon})
%     \end{aligned}
%   \end{equation}
%   is differentiable and approximate the $\min$ function as $\varepsilon \to 0$,
%   where $\ubar{z} = \argmin \mathbf{z}$.
% \end{definition}

% \begin{remark}[]
%   \cref{eqn:log-sum-exp-matrix-form} can be also written as a Softmin function as in \citet[Remark 4.22]{peyre2020}.
%   % , the Softmin function will approximate the $\min$ function and is differentiable.
%   % What is more, we can use the ``log-sum-exp'' trick to make this computation stable for any (arbitrarily small) $\varepsilon$.
%   However, here I choose \cref{eqn:log-sum-exp-matrix-form} as a more compact matrix notation.
%   This will also simplify the notation for calculating the jacobian matrices and gradients.
% \end{remark}

% In the case of Vanilla Sinkhorn in \cref{subsec:vanilla-sinkhorn} and Parallel Sinkhorn in \cref{subsec:parallel-sinkhorn},
% we use the \cref{eqn:sinkhorn-solution} to compute the optimal coupling matrix.
% Since we avoid calculating $\mathbf{u}$ and $\mathbf{v}$ in the log-stabilized Sinkhorn altogether,
% we need the solution $\mathbf{P}$ expressed by the dual variables $\mathbf{f}$ and $\mathbf{g}$.
% By \cref{eqn:sinkhorn-solution}, we have




Therefore, we can have the Log-stabilized Sinkhorn algorithm.

\begin{algorithm}[H]
  \caption{Log-Stabilized Sinkhorn Algorithm}
  \begin{algorithmic}[1]\label{algo:log-sinkhorn}
    \Require $\mathbf{a} \in \Sigma_M$, $\mathbf{b} \in \Sigma_N$, $\mathbf{C} \in \mathbb{R}^{M\times N}$, $\varepsilon > 0$.
    \Initialize $\mathbf{f} = \mathbb{0}_M$, $\mathbf{g} = \mathbb{0}_N$.
    \While{Not Convergence}
    \State \# update $f$
    \State $\mathbf{R} = \mathbf{C} - \mathbf{f}\cdot \mathbb{1}_N^\top - \mathbb{1}_M\cdot \mathbf{g}^\top$
    \State $c = \min \mathbf{R}$
    \State $\mathbf{Q} = \exp \left(
      -\frac{\mathbf{R} - c}{\varepsilon}
      \right)$
    \State $\mathbf{f} = \mathbf{f} + \varepsilon\log \mathbf{a} + c
      - \varepsilon\log \left[
        \mathbf{Q} \cdot\mathbb{1}_N
        \right]$
    \State \# update $g$
    \State $\mathbf{R} = \mathbf{C} - \mathbf{f}\cdot \mathbb{1}_N^\top - \mathbb{1}_M\cdot \mathbf{g}^\top$
    \State $c = \min \mathbf{R}$
    \State $\mathbf{Q} = \exp \left(
      -\frac{\mathbf{R} - c}{\varepsilon}
      \right)$
    \State $\mathbf{g} = \mathbf{g} + \varepsilon \log\mathbf{b} + c
      -\varepsilon\log \left[
        \mathbf{Q}^\top
        \cdot \mathbb{1}_M
        \right]$
    \EndWhile
    \State $\mathbf{P} = \exp \left(
      - \frac{\mathbf{C} - \mathbf{f}\,\mathbb{1}_N^\top - \mathbb{1}_M\,\mathbf{g}^\top}{\varepsilon}
      \right)$
    \Ensure $\mathbf{P}$
  \end{algorithmic}
\end{algorithm}

\begin{remark}[]
  The advantage of using log-stabilized Sinkhorn is to make sure the computation is stable for any arbitrary $\varepsilon$.
  The disadvantage, however, is that we cannot easily parallel its computation over several margins as in
  \cref{subsec:parallel-sinkhorn}.
  Moreover, the computation of matrix $\mathbf{Q}$ involves $\exp$ and $\log$ at every iteration twice
  for both updating $\mathbf{f}$ and $\mathbf{g}$.
  % Also, we include the softmin computation ($\exp$ and $\log$) at each iteration,
  % instead of the simple matrix multiplications,
  Thus, the log-stabilized Sinkhorn is less efficient than Vanilla and Parallel Sinkhorn algorithms.
  This is the tradeoff we need to make if we need to make sure the computation is stable,
  especially if we want to use a very small regularization constant $\varepsilon$.
  As in \citet[Chapter 4.1]{peyre2019},
  the Sinkhorn problem converges to original Kantorovich problem when $\varepsilon$ is small,
  i.e. $L_{\mathbf{C}}^\varepsilon(\mathbf{a},\mathbf{b}) \to L_{\mathbf{C}}(\mathbf{a},\mathbf{b})$
  as $\varepsilon \to 0$.
\end{remark}

\begin{remark}[Convergence Condition]\label{remark:conv-cond-parallel}
  Recall that in \cref{eqn:function-P}, we can recover the optimal coupling matrix from $\mathbf{f}$ and $\mathbf{g}$,
  then for some $\rho > 0$,
  \begin{equation*}
    \begin{aligned}
      \lVert
      \mathcal{P}\left(\mathbf{f}^{(\ell)}, \mathbf{g}^{(\ell)}\right) - \mathbf{a}
      \rVert_2 \le \rho
      \quad\text{and}\quad
      \lVert
      \mathcal{P}^\top \left(\mathbf{f}^{(\ell)}, \mathbf{g}^{(\ell)}\right) - \mathbf{b}
      \rVert_2 \le \rho.
    \end{aligned}
  \end{equation*}
\end{remark}
